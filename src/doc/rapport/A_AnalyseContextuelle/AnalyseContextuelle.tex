%-*- coding: utf-8 -*-
\textcolor[RGB]{46, 116, 181}{\chapter{Analyse Contextuelle}}
L'arrosage d'un jardin ou de plantes dans un espace de vente est une tâche qui doit être faite quotidiennement du printemps à l'automne.
Quand elle n'est pas faite manuellement, c'est un réseau de tuyaux d'arrosages proportionnel au nombre de plantes qui est mis en oeuvre.
Ce réseau doit être modifié à chaque fois que l'on déplace 1 ou plusieurs plantes. Il doit être vérifié et remis en état à chaque début de saison.
Le système proposé ici permet l'arrosage des plantes de la même façon que le ferait un être humain; sans réseau de tuyaux.

\jerome{\chapter{Problème} 
Le réseau de tuyaux doit être entretenu impose des coûts proportionnels à sa taille. 
Cet entretien doit être fait plusieurs fois par semaine, et l'arrosage des plantes efectué quotidiennement. Il faut donc embaucher du personnel qualifié 
pour répondre au problème du temps qui, là aussi, est proportionnel à la taille de la pépinière ou du jardin. }

\jeanfelix{Contribution de Jean-Félix.}

\samir{Contribution de Samir.}

\jerome{Contribution de Jérôme.}

\ela{Contribution de Ela.}

\nicolas{Contribution de Nicolas.}

\section{Finalité}
La finalité du système d'arrosage est de veiller à la bonne santé des plantes partout où c'est nécessaire.

\jerome{
Le système proposé contribue à l'entretien des espaces privés et au développement de la faune et de la flore.}

\section{Conceptualisation}
Les missions du système d'arrosage sont:
\begin{itemize}
\item Arroser les plantes aussi souvent que nécessaire.
\item Détecter la présence d'anomalies: parasites, maladies.

\jerome{
\item Arrosage automatique des pépinières et jardins
\item Parcourir une zone non cartographiée
\item Détecter et alerter les problèmes techniques liés au système}

\end{itemize}
