%-*- coding: utf-8 -*-
\textcolor[RGB]{46, 116, 181}{\chapter{Analyse Contextuelle}}
L'arrosage d'un jardin ou de plantes dans un espace de vente est une tâche qui doit être faite quotidiennement du printemps à l'automne.
Quand elle n'est pas faite manuellement, c'est un réseau de tuyaux d'arrosages proportionnel au nombre de plantes qui est mis en oeuvre.
Ce réseau doit être modifié à chaque fois que l'on déplace 1 ou plusieurs plantes. Il doit être vérifié et remis en état à chaque début de saison.
Le système proposé ici permet l'arrosage des plantes de la même façon que le ferait un être humain; sans réseau de tuyaux.

\section{Finalité}
La finalité du système d'arrosage est de veiller à la bonne santé des plantes partout où c'est nécessaire.

\section{Conceptualisation}
Les missions du système d'arrosage sont:
\begin{itemize}
\item Arroser les plantes aussi souvent que nécessaire.
\item Détecter la présence d'anomalies: parasites, maladies.
\end{itemize}
